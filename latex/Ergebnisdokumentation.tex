\documentclass[a4paper,11pt,listof=numbered,glossary=totoc,parskip=half,toc=bib]{scrreprt}
\usepackage[a4paper,left=2cm,right=2cm,top=2cm,bottom=2cm]{geometry}
\usepackage[onehalfspacing]{setspace}

\usepackage[colorlinks,
pdfpagelabels,
pdfstartview = FitH,
bookmarksopen = true,
bookmarksnumbered = true,
linkcolor = black,
plainpages = false,
hypertexnames = false,
citecolor = black]{hyperref}

\usepackage[utf8]{inputenc}
\usepackage[T1]{fontenc}
\usepackage[ngerman]{babel}
\usepackage{graphicx}
\usepackage{caption}
\usepackage[automake,toc,section=chapter,numberedsection]{glossaries}
\usepackage{uarial}
\usepackage{tabularx}
\usepackage{booktabs}
\usepackage{multirow}
\usepackage{icomma} % Damit im Mathemodus nach einem Komma kein Leerzeichen gesetzt wird
\usepackage[cache=false]{minted}
\renewcommand{\listoflistingscaption}{Verzeichnis der Code-Listings}
\usepackage[table]{xcolor} % Zum Ändern der Farben in Tabellen
\usepackage{xspace}
\usepackage{appendix}

\usepackage{csquotes}
\usepackage[backend=biber, style=apa]{biblatex}
\addbibresource{quellen.bib}

\renewcommand{\familydefault}{\sfdefault}
\newcommand{\zB}{\mbox{z.\,B.}\xspace}
\newcommand{\dash}{\mbox{d.\,h.}\xspace}

\RedeclareSectionCommand[
beforeskip = 3pt,
afterskip = 3pt]{subsection} %vor subsection 6pt und nach subsection 6pt Abstand
\RedeclareSectionCommand[
beforeskip = 1pt,
afterskip = 1pt]{subsubsection} %vor subsection 6pt und nach subsection 6pt Abstand


% Glossar
\makeglossary
\newglossaryentry{re}
{
	name={Requirements Engineering (RE)},
	description={Das \textit{Requirements Engineering} bezeichnet die Disziplin der Anforderungsermittlung. Damit ist typischerweise das Ermitteln, Dokumentieren, Prüfen und Abstimmen von funktionalen und nicht-funktionalen Anforderungen gemeint.}
}

\newglossaryentry{stakeholder}
{
	name={Stakeholder},
	description={\textit{Stakeholder} (dt. Anspruchsgruppen) sind alle Personen, die mit dem zu entwickelnden System konfrontiert sind.
Der Begriff beschränkt sich nicht nur auf diejenigen, die unmittelbar mit dem System arbeiten, sondern schließt insbesondere den Auftraggeber, das Entwicklerteam oder die mit dem Betrieb des Systems betrauten Personen mit ein.}
}
\newglossaryentry{userstory}
{
	name={User Story},
	description={Eine \textit{User Story} ist eine besonders in agilen Projekten weit verbreitete Dokumentationsform im Kontext der Anforderungsermittlung.
Sie besteht aus einem einfachen Satz, der eine Anforderung aus der Sicht einer definierten Stakeholder-Rolle beschreibt und entspricht immer einem fest vorgegebenen Format:
\textit{Als [Rolle] möchte ich / wünsche ich mir [Funktion], damit [Begründung].}
User Stories lassen sich damit auf einfache Karteikarten schreiben, die an ein Whiteboard gepinnt oder auf einem großen Tisch ausgebreitet werden können. Somit lassen sie sich gut in Workshops zur Visualisierung und Priorisierung der Anforderungen nutzen.}
}
\newglossaryentry{moscow}
{
	name={MoSCoW},
	description={Die \textit{MoSCoW-Methode} wird im vorliegenden Projekt genutzt, um Anforderungen zu priorisieren.
\textit{Must have (M)} legt fest, dass die betrachtete Anforderung zwingend umgesetzt werden muss.
\textit{Should have (S)} bedeutet, dass die Umsetzung wünschenswert, aber nicht kritisch ist.
\textit{Can have (C)} bezeichnet unkritische Anforderungen, die optional zu einem späteren Zeitpunkt implementiert werden können.
\textit{Won't have (W)} legt fest, dass die betrachtete Anforderung (aktuell) nicht umgesetzt wird.
Die Priorisierung von Anforderungen ist nicht endgültig, sondern kann im Projektverlauf angepasst werden.
	}
}
\newglossaryentry{git} 
{
	name={Git-Repository},
	description={GIT ist ein Werkzeug zur Versionskontrolle, dass vor allem zur kollaborativen Quellcode-Verwaltung in Software-Projekten eingesetzt wird, sich aber auch zur Verwaltung und Versionierung von Artefakten der Dokumentation eignet.
Die von Microsoft betriebene Plattform GitHub.com bietet einen Dienst für das Hosting von Git-Repositories.}
}
\newglossaryentry{responsive}
{
	name={Responsive Design},
	description={Ist eine Applikation so gestaltet, dass sie sich an verschiedene Bildschirmgrößen und -ausrichtungen, wie sie beispielsweise bei Smartphones-, Tablets oder Desktop-PCs zu finden sind, anpassen kann, so spricht man von \textit{Responsive Design}.
	}
}
\newglossaryentry{uml}
{
	name={UML},
	description={Die \textit{Unified Modeling Language} ist eine grafische Modellierungssprache zur Analyse, Implementation und zum Design von softwarebasierten Systemen sowie zur Beschreibung von Prozessen. \autocite{UML} Sie wird durch die \textit{Object Management Group} entwickelt. Die Sprache definiert jeweils sieben Struktur- und Verhaltensdiagramme.
	In diesem Bericht werden folgende Diagrammtypen genutzt: 	\textit{Komponentendiagramm (cod)},\textit{ Klassendiagramm (cld)}, \textit{Zustandsdiagramm (sm)} und \textit{Sequenzdiagramm (sqd)}.
	}
}
\newglossaryentry{bpmn}
{
	name={BPMN},
	description={Die \textit{Business Process Model and Notation} ist ein Standard zur grafischen Spezifikation von Geschäftsprozessen. \autocite{BPMN} 
	}
}
\newglossaryentry{erm}
{
	name={ERM},
	description={Ein \textit{Entity-Relationship-Model} ist ein Modell zur Darstellung von Entitäten und Beziehungen. Der Einsatz von ERM gilt als Standard bei der Datenmodellierung in der Softwareentwicklung.
	}
}
\newglossaryentry{gui}
{
	name={GUI},
	description={Grafische Benutzeroberfläche
	}
}

\newglossaryentry{ssltls}
{
	name={SSL/TLS},
	description={Secure Socket Layer / Transport Layer Security. Verfahren zur Verschlüsselung von Netzwerkverbindungen.
	}
}

\newglossaryentry{ssh}
{
	name={SSH},
	description={Secure Shell. Verschlüsseltes Netzwerkprotokoll zum Konsolenzugriff auf Serversysteme.
	}
}


\subject{Meilensteinbericht}
\title{Meilenstein 3}
\subtitle{Dokumentationskonzept bereitgestellt}

\begin{document}
	\pagenumbering{Roman}
	\begin{titlepage}
		
		\centering
		\vspace*{2.5cm}
		{\large\bfseries \par}	
		{\Huge\bfseries Ergebnisdokumentation\par}
		{\Large\bfseries  \par}

		{\Large Projekt Q-Teams\par}
		{\large\today\par}
		\vspace{0.5cm}

			
		
		\includegraphics[scale=0.5]{iubh_logo}
		
		IUBH Fernstudium
		\vspace{0.5cm}
		
		\begin{tabular}{lllrl}
			\toprule
			\textbf{Gruppe} & \textbf{Nachname} & \textbf{Vorname} & \textbf{Matrikelnr.} & \textbf{Studiengang} \\
			\midrule
			Projektleiter & Sawatzki & Jörg & 9186524 & BSc. Informatik \\
			Mitglied 2 & Hahn & Maximilian & 91710055 & BSc. Wirtschaftsinformatik \\
			Mitglied 3 & Lapenat & Holger & 3191237 & BSc. Wirtschaftsinformatik \\
			Mitglied 4 & Moch & Daniel & 91710824 & BSc. Wirtschaftsinformatik \\
			\bottomrule
		\end{tabular}	
	\end{titlepage}
	
	
	\newpage
	\setcounter{tocdepth}{2}
	\tableofcontents	
	\renewcommand \thechapter{\Roman{chapter}}
	\listoffigures % ABBILDUNGSVERZEICHNIS
	\printglossaries
	\newcounter{lastRomanCounter}
	\setcounter{lastRomanCounter}{\value{chapter}} % Zwischenspeicher der Kapitelnummerierung (römisch)
	
	\newpage
	\renewcommand \thechapter{\arabic{chapter}}
	\pagenumbering{arabic}	
	\setcounter{chapter}{0}
	
	\chapter{Einleitung}
	
	\chapter{Funktionsbeschreibung}
	Mit dem Prototyp ist das Spielen Spielern im Wettkampf- und Trainingsmodus möglich. Der Spielablauf des Protoyps ist in Abbildung \ref{fig:bpmn} gezeigt. 
	Die Darstellung zeigt exemplarisch den Ablauf für einen Spieler des Teams. Das Spiel kann mit zwei oder mehr Spielern gespielt werden. 
		
	\begin{figure}
		\centering
		\includegraphics[width=\textwidth]{bpmn_IST.png}
		\caption{Der Spielablauf}
		\label{fig:bpmn}
	\end{figure}
	
	\section{Beginn des Spiels}
	Ein Spieler gründet ein Q-Team und legt dabei Teamnamen und Thema dauerhaft fest. Er wird zum Teamleiter. Der Teamname ist frei wählbar. Das Thema ist eines der Module der IUBH, diese sind mit Kennung und Namen gespeichert (\zB \textit{IGIS01 Grundlagen der industriellen Softwaretechnik}). 
	Wird mit einem bereits bestehenden Team eine weitere Runde gespielt, ist hier der Einstiegspunkt für die weitere Runde.
	An diesem Punkt kann der Teamleiter Spieler hinzufügen.
	Zuletzt legt er fest, ob im Wettkampf- oder im Trainingsmodus gespielt werden soll und startet die Runde.
	Im Wettkampfmodus sind die Statistiken der Mitspieler für jeden jederzeit sichtbar (\dash die Anzahl an richtig, teilweise richtig und falsch beantworteten Fragen, sowie die Punktzahl). Im Trainingsmodus ist lediglich die eigene Statistik sichtbar. Die Statistik ist in den weiteren Phasen jederzeit sichtbar und wird zu Beginn einer Runde zurückgesetzt.
	
	Mitspieler, die einem Team hinzugefügt wurden rufen lediglich das Team auf und warten auf den Start durch den Teamleiter.
	
	\section{Erarbeitungsphase}
	In der Erarbeitungsphase erstellt jeder Spieler eine Frage und die dazugehörige Musterantwort. Die
Knowledge Base, d. h. die Gesamtheit der bisher gespeicherten Fragen und Musterantworten, kann
bei dieser Aktivität als Inspiration genutzt werden. Die durch den Spieler erstellte Frage und Antwort
werden in der Knowledge Base und im Katalog der Runde gespeichert. 
Parallel zum Erstellen einer Frage können die Spieler die bereits in der Knowledge Base gespeicherten Fragen nach Thema sortiert durchsuchen.
Nachdem alle Spieler ihre Frage abgesendet haben, beginnt die erste Fragephase automatisch.

	\section{Fragephase}
	Es wird automatisch die nächste noch nicht beantwortete Frage gewählt. Der Fragesteller hat in dieser Phase keine Aufgabe. Alle anderen Spieler beantworten die Frage und speichern ihre Antwort ab. Jede Antwort wird im Katalog der Runde abgespeichert. Sind alle Antworten gespeichert geht das Spiel automatisch in die Bewertungsphase über.
	
	\section{Bewertungsphase}
	Des Spielern werden nun die Frage, die Musterantwort und alle Spielerantworten angezeigt. Der Fragesteller nimmt nun die Bewertung der Spielerantworten vor und vergibt somit die Punkte (richtig 3 Punkte, teilweise richtig 1 Punkt, falsch 0 Punkte). Die Mitspieler sehen die Bewertungen der Fragestellers sofort. Sollte also Klärungsbedarf bestehen, können sich die Spieler noch in dieser Phase austauschen. Erst nachdem der Fragesteller die Bewertungsphase abschließt, werden die Bewertungen gespeichert, vorher kann der Fragesteller diese ändern. Ist die Phase abgeschlossen, wird geprüft, ob noch nicht beantwortete Fragen im Katalog der Runde vorhanden sind. Ist dies der Fall schließt sich nun eine weitere Fragephase an, sonst geht das Spiel in die Schlussphase über.
	
	\section{Schlussphase}
	In dieser Phase wird weiterhin die Statistik der Runde gezeigt. Der Teamleiter kann nun eine weitere Runde vorbereiten und starten oder das Spiel beenden.
	
	\section{Das Q-Team}
	Nachdem das Q-Team gegründet wurde besteht es nur aus dem Teamleiter. Sobald der Teamleiter mindestens einen weiteren Spieler zum Team hinzugefügt hat, ist das Team spielbereit. Zu Beginn des Spiels und in der Schlussphase kann der Teamleiter Spieler hinzufügen und entfernen, außerdem können die Spieler das Team verlassen.
	
	Die Rolle des Teamleiters kann nicht wechseln, \dash der Teamleiter kann das Team erst als letzter Spieler verlassen, das Team wird in diesem Fall gelöscht.
	
	Das Team besteht über eine Runde hinaus. Nur der Teamleiter kann durch Verlassen des Teams dieses löschen.
	
	\chapter{Softwarearchitektur}
	\section{Systemstruktur}
		\subsection{Backend}
	\label{subsec:backend}
	Das serverseitige Backend übernimmt die Datenhaltung sowie die fachliche Anwendungslogik und stellt die Schnittstelle zum Frontend bereit. Es wurde in der Programmiersprache Python mit dem Django-Framework\footnote{https://www.djangoproject.com/} implementiert. Django ist ein auf einer \frqq{}Model View Controller (MVC)\flqq{}-Architektur basierendes Framework für datenbankgestützte Web-Applikationen.
	
	Das Django-Framework abstrahiert über einen Object Relational Mapper (ORM) den Datenbankzugriff, so dass verschiedene relationale Datenbanken ohne Anpassung des Programmcodes verwendet werden können. Da es sich bei diesem Projekt lediglich um einen Prototyp handelt, wurde das SQLite-Backend gewählt. SQLite\footnote{https://sqlite.org} ist ein dateibasiertes Datenbanksystem, das ohne einen Datenbankserver auskommt und sich deshalb schnell und unkompliziert integrieren lässt.
	
	Zur Anbindung des Frontends stellt der Server eine GraphQL-Schnittstelle bereit. GraphQL\footnote{https://graphql.org/} ist eine von Facebook entwickelte schemabasierte Abfragesprache. Im Gegensatz zu klassischen REST-APIs erlaubt GraphQL dem Client genau festzulegen, welche Datenfelder und -beziehungen vom Server zurück geliefert werden, so dass das Entwerfen benutzerdefinierter Query-Parameter nicht mehr nötig ist. Darüber hinaus wird mithilfe des GraphQL-Schemas bereits auf dem Client verhindert, dass fehlerhafte Anfragen formuliert werden können.
	
	GraphQL ermöglicht es mit \textit{Subscriptions} einem Client, sich über Ereignisee benachrichtigen zu lassen. Dieser Mechanismus ist mit den von mobilen Geräten bekannten Push-Benachrichtigungen vergleichbar. Q-Teams verwendet Subscriptions, um den Spielablauf auf den Clients aller beteiligten Spieler zu synchronisieren. Technisch realisiert wird dies über eine WebSocket-Verbindung.

	Der Backend-Server stellt außerdem eine serverseitig gerenderte Web-Oberfläche zur Administration der Q-Teams-Instanz bereit. Diese basiert auf der angepassten Django-Administrationsoberfläche.
	
	\subsection{Frontend}
	Das Frontend stellt die grafische Oberfläche dar, mit der die Spieler interagieren. Sie wird im Webbrowser ausgeführt und wurde dementsprechend in TypeScript implementiert. TypeScript ergänzt JavaScript um statische Typprüfung. Damit können typbedingte Fehler, die sonst oft erst zur Laufzeit sichtbar werden, schon während des Kompiliervorgangs erkannt und beseitigt werden. 
	
	 Für die Umsetzung der \Gls{gui}-Komponenten kam das React-Framework\footnote{https://reactjs.com} zum Einsatz. React ist ein von Facebook veröffentlichtes, deklaratives und komponentenorientiertes Framework zur schnellen Entwicklung von clientseitig gerenderten Benutzerschnittstellen. Für die Kommunikation mit dem Backend per GraphQL kommt \textit{Relay}\footnote{https://relay.dev/} zum Einsatz.
	
	Details zur Anbindung an das Backend wurden bereits im Abschnitt \ref{subsec:backend} beschrieben.	
	
	\subsection{Komponenten}
	
	
	\begin{figure}
		\centering
		\includegraphics[width=\textwidth]{deployment}
		\caption{Systemkomponenten und Schnittstellen}
		\label{fig:components}
	\end{figure}
	
	Abbildung \ref{fig:components} zeigt die Systemkomponenten, die im folgenden beschrieben werden.
	
	\begin{itemize}
		\item Die \textit{SQLite}-Datenbank stellt die Datenhaltung dar. Sie speichert die fachlichen Entitäten und Beziehungen in Tabellen.
		\item Das mit dem \textit{Django-Framework} entwickelte Backend implementiert die Spiellogik und steuert den Benutzerzugriff. Es greift über den Object Relational Mapper (ORM) auf die Datenbank zu.
		\item Der \textit{Daphne Application Server} stellt die Laufzeitumgebung für das Django-Backend bereit. Er kommuniziert mit dem Backend über das Asynchronous Server Gateway Interface (ASGI), einem Standard zur Server-Anbindung asynchroner Python-Web-Apps.
		\item Der vom Hoster Uberspace bereitgestellte Apache-Webserver bindet als Reverse Proxy das Backend an das öffentliche Internet an. Er kommuniziert über HTTPS bzw. verschlüsselte WebSockets (WSS) mit dem Application Server und stellt statische Assets (HTML, CSS, JS, Bilder) direkt aus dem Dateisystem zur Verfügung.
		\item Das React-Frontend wird über HTTPS vom Server geladen und im Browser ausgeführt. Es kommuniziert per GraphQL mit dem Backend und nutzt dabei sowohl das HTTPS-, als auch das WSS-Protokoll.
	\end{itemize}
	
	
	\section{Daten}
	
	\begin{figure}
		\centering
		\includegraphics[width=\textwidth]{models_final}
		\caption{Finale Datenbankstruktur}
		\label{fig:database}
	\end{figure}

	Abbildung \ref{fig:database} zeigt die finale Datenbankstruktur der Applikation.
	
	Die bereits im Bericht zu Meilenstein 3 vorgestellte Grundstruktur ist deutlich zu erkennen, allerdings wurde das Modell im Verlauf des erkenntnisgetriebenen Entwicklungsprozesses weiter verfeinert und angepasst.
	
		\subsection{Team}
	Die Klasse \textit{Team} ist das Herzstück der Quiz-App. Sie bildet ein Team von Benutzern ab, die zusammen spielen. Außerdem speichert sie den Benutzer, der das Team erstellt hat (\textit{creator}), enthält einen frei wählbaren Namen (\textit{name}) und wird einem Thema (\textit{topic}) zugeordnet.
	
	Das Feld für den Spielmodus (\textit{mode}) kann die Werte \texttt{train} (Trainingsmodus) oder \texttt{competition} (Wettkampf) annehmen.
	
	Ein Team befindet sich immer in einem definierten Zustand (\textit{state}):
	
	\begin{itemize}
		\item \texttt{open}: Das Team wurde erstellt und kann Mitglieder aufnehmen.
		\item \texttt{question}: Diese Phase stellt die \textit{Erarbeitungsphase} dar. In dieser Phase erarbeiten die Teammitglieder fachliche Fragen für ihre Mitspieler. 
		\item \texttt{answer}: In dieser Phase müssen die Spieler die erarbeiteten Fragen beantworten. Das System wählt die jeweils zu bearbeitende Frage zufällig aus und hinterlegt sie im Feld \textit{current\_{}question}.
		\item \texttt{scoring}: Bei Eintritt in die \textit{Bewertungsphase} haben die Spieler ihre Antworten eingegeben und es erfolgt die Bewertung durch den Fragesteller. Sind noch unbeantwortete Fragen in der Runde vorhanden, so wechselt die Runde wieder in den Zustand \texttt{answer}.
		
		\item \texttt{done}: Diese Phase markiert den Abschluss der Runde. Spielern wird abhängig vom gewählten Spielmodus nur eine persönliche oder eine gesamte Statistik der Leistung des Teams angezeigt. In dieser Phase können Mitglieder hinzugefügt oder entfernt werden und der Modus gewechselt werden. Das Team wechselt ausgehend von dieser Phase wieder in die Phase \textit{question}, falls eine neue Runde gespielt werden soll.
		
	\end{itemize}
	
	\subsection{Membership}
	Die Klasse \textit{Membership} bildet die Mitgliedschaft eines Benutzers (\textit{user}) in einem Team \textit{team} ab.
	Die Attribute \textit{right}, \textit{wrong} und \textit{partial} speichern die Anzahl der Fragen, die der Nutzer in der aktuellen Runde richtig, falsch bzw. teilweise richtig beantwortet hat. 
	
	\subsection{Question}
	Die Klasse \textit{Question} bildet eine fachliche Frage ab. Sie speichert den Autor (\textit{author}) der Frage, den Fragetext (\textit{question}), das zugehörige Thema (\textit{topic}), die Musterantwort (\textit{model\_{}answer}) und eine Referenz auf das Team.
	
	\subsection{Answer}
	Die Klasse \textit{Answer} enthält die Antwort (\textit{answer}) eines Spielers (\textit{author}) auf eine Frage (\textit{question}). Sie speichert außerdem die durch den Fragesteller vergebene Bewertung (\textit{score}).
	
	\subsection{Topic}
	Die Klasse \textit{Topic} stellt ein fachliches Thema dar. Hier wurde beispielhaft eine Liste der an der IUBH verfügbaren Module hinterlegt werden. Das Attribut \textit{name} speichert die Kursbezeichnung, das Attribut \textit{code} den Modulcode. Diese Klasse dient der besseren Strukturierung der Fragen in der \textit{Knowledge Base}.

	\subsection{User}
	Die Klasse \textit{User} entspricht dem unveränderten Benutzermodell des Django-Frameworks. Details dazu finden sich in der Django-Dokumentation\footnote{https://docs.djangoproject.com/en/3.0/ref/contrib/auth/}.	
	
	\section{GUI}
	
	\begin{figure}
		\centering
		\includegraphics[width=\textwidth]{gui.png}
		\caption{GUI-Masken und Übergänge}
		\label{fig:gui}
	\end{figure}
	
	Dieser Abschnitt beschreibt in Kürze den Aufbau und das Zusammenspiel der GUI-Masken der App. Für weitere Details sei auf das Benutzerhandbuch verwiesen.
	
	Das Zusammenspiel der GUI-Masken in der App ist in Abbildung \ref{fig:gui} schemenhaft dargestellt. Die mit Sternchen (*) markierten Masken sind über die Navigation der App direkt erreichbar und somit kann ein Wechsel zur entsprechenden GUI-Maske über einen Klick auf den Link von jeder beliebigen Stelle der App ausgelöst werden -- diese Übergänge wurden aus Gründen der Übersichtlichkeit in der Grafik nicht eingezeichnet.
	
	Auch die Masken für Login und Logout werden an dieser Stelle ausgelassen.
	
	Die grafische Oberfläche gliedert sich in drei Hauptbereiche:
		
	\subsection{Home}
	Der Bereich \textit{Home} ist die Landing Page des Q-Teams-Projektes. Sie stellt mithilfe des Projektvideos das Konzept vor.

	\subsection{Spielen}
	Im Bereich \textit{Spielen} findet sich eine Übersicht über die Teams, in denen der Benutzer Mitglied ist. Darüber hinaus kann er an dieser Stelle eigene Teams erstellen.
	
	Hat er ein Team erstellt oder wählt eines aus, so gelangt er zur Maske \textit{Teamspiel}
	
	\subsubsection{Teamspiel}	

	\begin{figure}
		\centering
		\includegraphics[width=\textwidth]{ablauf}
		\caption{Phasen des Spiels}
		\label{fig:phasen}
	\end{figure}
		
	Diese Seite ist der Kern der Applikation, denn dort findet das eigentliche Quiz statt. Abhängig von der jeweiligen Phase des Spiels sind an dieser Stelle verschiedene Funktionen verfügbar. Ablauf und Funktionen sind in Abbildung \ref{fig:phasen} skizziert.
	
	
	\subsection{Knowledge Base}
	Die \textit{Knowledge Base} stellt den Zugang zu bereits gespielten Fragen aller Spieler bereit und dient als Inspiration für eigene fachliche Fragen.
	
	Auf der Übersichtsseite der Knowledge Base kann der Nutzer einen Suchbegriff eingeben oder ein Thema wählen. Sodann gelangt er auf die Seite \textit{Knowledge Base Browser}.
	
	\subsubsection{Knowledge Base Browser}
	Diese Seite gibt das Suchergebnis mit den gefundenen Fragen und Musterantworten aus und bietet weitere Filtermöglichkeiten.
	
	Ist der aktuelle Nutzer auch der Autor einer Frage, so hat er die Möglichkeit, diese zu editieren oder zu löschen. Nach einem Klick auf den entsprechenden Button gelangt er dazu auf die Seite \textit{Frage editieren} bzw. \textit{Frage löschen}.

	\setcounter{chapter}{0}
	
	
	
	% ANHANG
	\begin{appendices}
	\renewcommand{\appendixtocname}{Anhang}
	\renewcommand{\appendixpagename}{Anhang}
	\appendixpage	
	\addappheadtotoc
	\chapter{Anforderungen}	
		
		\section{Funktionale Anforderungen}
		
	\subsection{Als Spieler möchte ich mit dem Quizsystem gezielt fachliche Inhalte wiederholen, um mich auf
Prüfungen vorzubereiten.}

		\begin{tabularx}{\textwidth}{Xr}
			
			Kriterien & Erfüllungsgrad \\
			\midrule
		Das Thema einer Quizrunde kann vor Beginn festgelegt werden. & \\
		Die Module der IUBH können als Thema ausgewählt werden. & \\
		Die Lernzyklen der Module können als Thema ausgewählt werden. & Nicht umgesetzt \\
			\bottomrule
		\end{tabularx}		
		
		\subsection{Als Spieler möchte ich gegen einen anderen Spieler antreten können, um mich mit ihm zu messen /
vergleichen.}

		\begin{tabularx}{\textwidth}{Xr}
			
			Kriterien & Erfüllungsgrad \\
			\midrule
		Das Spiel kann mit anderen Menschen gespielt werden. & \\
		Es kann ein kompetitiver Spielmodus festgelegt werden. & \\
		Den Spielern werden ceteris paribus die gleichen Fragen gestellt. & \\		
		Die Antworten werden mit Punkten bewertet. & \\
		Den Spielern wird eine Rangliste anhand ihrer Punkte angezeigt. & \\
			\bottomrule
		\end{tabularx}	
		
		\subsection{Als Spieler möchte ich mit einem anderen Spieler gemeinsam spielen, um mich im Team über Lerninhalte austauschen zu können.}	
			\begin{tabularx}{\textwidth}{Xr}
			
			Kriterien & Erfüllungsgrad \\
			\midrule
		Das Spiel kann mit anderen Menschen gespielt werden. & \\
		Es kann ein Team gebildet werden. & \\
		Es kann ein kooperativer Spielmodus festgelegt werden. & \\
			\bottomrule
		\end{tabularx}	
		
		\subsection{Als Spieler möchte ich Fragen und Antworten für andere Spieler hinzufügen und pflegen, um mein
Wissen weiterzugeben und zu festigen.}
		\begin{tabularx}{\textwidth}{Xr}
			
			Kriterien & Erfüllungsgrad \\
			\midrule
		Fragen und Antworten können persistiert werden. & \\
		Persistierte Fragen und Antworten können angepasst werden. & \\
		Persistierte Fragen und Antworten können gelöscht werden. & \\
		Persistierte Fragen und Antworten können angesehen werden. & \\
		Persistierte Fragen und Antworten können in einer Spielrunde genutzt werden. & \\
			\bottomrule
		\end{tabularx}		
		
		\subsection{Als Spieler möchte ich Feedback zu meiner Lösung bekommen, um aus Fehlern zu lernen.}
		\begin{tabularx}{\textwidth}{Xr}
			
			Kriterien & Erfüllungsgrad \\
			\midrule
			Die Antworten werden bewertet. & \\
			Dem Spieler wird die Bewertung der Antwort angezeigt. & \\
			
			\bottomrule
		\end{tabularx}	
		
		\section{Nicht-funktionale Anforderungen}
		
		\subsection{Als Spieler möchte ich das Quiz im Browser spielen können, damit ich es auf verschiedenen
Plattformen spielen kann und keine Zusatzsoftware installieren muss.}
		\begin{tabularx}{\textwidth}{Xr}
			
			Kriterien & Erfüllungsgrad \\
			\midrule
		\textit{Spielen} bedeutet hier den Ablauf einer Spielrunde in den Modi \textit{kompetitiv} und \textit{kooperativ} in einem Team mit 5 Spielern ohne Spielabbrüche durchzuspielen. 	
		Das Spiel ist mit folgenden Browsern in der Standardinstallation spielbar: & \\
		Quizrunde mit folgenden Browsern spielen: & \\
		IE 11.778.18362.0 & \\
		Firefox 76.0 & \\
		Edge 81.0.416.68 & \\
		Chrome 81.0.4044.138 & \\
			\bottomrule
		\end{tabularx}	
			
		\subsection{Als Auftraggeber wünsche ich mir einen lauffähigen Prototyp, damit ich die mögliche Einsatzfähigkeit in der Praxis evaluieren kann.}
		\begin{tabularx}{\textwidth}{Xr}
			
			Kriterien & Erfüllungsgrad \\
			\midrule
			Es kann im Browser Firefox 76.0 in einem Team mit 3 Spielern jeweils eine Spielrunde in den Modi \textit{kompetitiv} und \textit{kooperativ} durchgespielt werden. & \\
			Es kann im Browser Firefox 76.0 in der Erarbeitungsphase eine Frage mit Antwort persistiert werden. & \\
			Es können im Browser Firefox 76.0 in der Erarbeitungsphase die bereits persistierten Fragen mit Antworten angezeigt werden. & \\
			
			\bottomrule
		\end{tabularx}	
				
		\subsection{Als Auftraggeber / Spieler wünsche ich mir ein Benutzerhandbuch, um mir einen Überblick über
Aufbau und Funktionen des Systems verschaffen zu können.}
		\begin{tabularx}{\textwidth}{Xr}
			
			Kriterien & Erfüllungsgrad \\
			\midrule
		Das Benutzerhandbuch wird dem Auftraggeber zugestellt. & \\
		Das Benutzerhandbuch steht einem Spieler in dem System zur Verfügung. & \\
		Jede GUI-Maske wird mit einem Screenshot dargestellt und erläutert. & \\
		Die Spielregeln werden erklärt. & \\
			\bottomrule
		\end{tabularx}	
		
		\subsection{Als Auftraggeber / Betreiber wünsche ich mir eine Dokumentation zur Installation und
Inbetriebnahme, damit ich das Quiz ohne großen Aufwand den Studenten zur Verfügung stellen
kann.}
		\begin{tabularx}{\textwidth}{Xr}
			
			Kriterien & Erfüllungsgrad \\
			\midrule
		Die Dokumentation wird dem Auftraggeber zugestellt. & \\
		Die Dokumentation steht einem potenziellen Betreiber zur Verfügung. & \\
		Die Dokumentation beschreibt die notwendige Hardware zur Installation und Inbetriebnahme. & \\
		Die Dokumentation beschreibt die notwendige Software zur Installation und Inbetriebnahme. & \\
		Die Dokumentation beschreibt die notwendigen Schritte zur Installation und Inbetriebnahme. & \\
		Die Dokumentation beschreibt mögliche Fehlerfälle mit den dazugehörigen Lösungen. & \\
		
			\bottomrule
		\end{tabularx}	

		
			
		
		
		
	
	\chapter{Qualitätsziele}
	\chapter{Benutzerhandbuch}
	\chapter{Administration}
	\chapter{Inbetriebnahme}
	
	\end{appendices}
	\newpage	
	\setcounter{chapter}{\thelastRomanCounter} % Fortsetzen der römischen Kapitelnummerierung
\renewcommand \thechapter{\Roman{chapter}}	\printbibliography[heading=bibnumbered,title=Literaturverzeichnis]

\end{document}
