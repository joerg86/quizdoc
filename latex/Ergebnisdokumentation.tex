\documentclass[a4paper,11pt,listof=numbered,glossary=totoc,parskip=half,toc=bib]{scrreprt}
\usepackage[a4paper,left=2cm,right=2cm,top=2cm,bottom=2cm]{geometry}
\usepackage[onehalfspacing]{setspace}

\usepackage[colorlinks,
pdfpagelabels,
pdfstartview = FitH,
bookmarksopen = true,
bookmarksnumbered = true,
linkcolor = black,
plainpages = false,
hypertexnames = false,
citecolor = black]{hyperref}

\usepackage[utf8]{inputenc}
\usepackage[T1]{fontenc}
\usepackage[ngerman]{babel}
\usepackage{graphicx}
\usepackage{caption}
\usepackage[automake,toc,section=chapter,numberedsection]{glossaries}
\usepackage{uarial}
\usepackage{tabularx}
\usepackage{booktabs}
\usepackage{multirow}
\usepackage{icomma} % Damit im Mathemodus nach einem Komma kein Leerzeichen gesetzt wird
\usepackage[cache=false]{minted}
\renewcommand{\listoflistingscaption}{Verzeichnis der Code-Listings}
\usepackage[table]{xcolor} % Zum Ändern der Farben in Tabellen
\usepackage{xspace}
\usepackage{appendix}

\usepackage{csquotes}
\usepackage[backend=biber, style=apa]{biblatex}
\addbibresource{quellen.bib}

\renewcommand{\familydefault}{\sfdefault}
\newcommand{\zB}{\mbox{z.\,B.}\xspace}
\newcommand{\dash}{\mbox{d.\,h.}\xspace}

\RedeclareSectionCommand[
beforeskip = 3pt,
afterskip = 3pt]{subsection} %vor subsection 6pt und nach subsection 6pt Abstand
\RedeclareSectionCommand[
beforeskip = 1pt,
afterskip = 1pt]{subsubsection} %vor subsection 6pt und nach subsection 6pt Abstand


% Glossar
\makeglossary
\newglossaryentry{re}
{
	name={Requirements Engineering (RE)},
	description={Das \textit{Requirements Engineering} bezeichnet die Disziplin der Anforderungsermittlung. Damit ist typischerweise das Ermitteln, Dokumentieren, Prüfen und Abstimmen von funktionalen und nicht-funktionalen Anforderungen gemeint.}
}

\newglossaryentry{stakeholder}
{
	name={Stakeholder},
	description={\textit{Stakeholder} (dt. Anspruchsgruppen) sind alle Personen, die mit dem zu entwickelnden System konfrontiert sind.
Der Begriff beschränkt sich nicht nur auf diejenigen, die unmittelbar mit dem System arbeiten, sondern schließt insbesondere den Auftraggeber, das Entwicklerteam oder die mit dem Betrieb des Systems betrauten Personen mit ein.}
}
\newglossaryentry{userstory}
{
	name={User Story},
	description={Eine \textit{User Story} ist eine besonders in agilen Projekten weit verbreitete Dokumentationsform im Kontext der Anforderungsermittlung.
Sie besteht aus einem einfachen Satz, der eine Anforderung aus der Sicht einer definierten Stakeholder-Rolle beschreibt und entspricht immer einem fest vorgegebenen Format:
\textit{Als [Rolle] möchte ich / wünsche ich mir [Funktion], damit [Begründung].}
User Stories lassen sich damit auf einfache Karteikarten schreiben, die an ein Whiteboard gepinnt oder auf einem großen Tisch ausgebreitet werden können. Somit lassen sie sich gut in Workshops zur Visualisierung und Priorisierung der Anforderungen nutzen.}
}
\newglossaryentry{moscow}
{
	name={MoSCoW},
	description={Die \textit{MoSCoW-Methode} wird im vorliegenden Projekt genutzt, um Anforderungen zu priorisieren.
\textit{Must have (M)} legt fest, dass die betrachtete Anforderung zwingend umgesetzt werden muss.
\textit{Should have (S)} bedeutet, dass die Umsetzung wünschenswert, aber nicht kritisch ist.
\textit{Can have (C)} bezeichnet unkritische Anforderungen, die optional zu einem späteren Zeitpunkt implementiert werden können.
\textit{Won't have (W)} legt fest, dass die betrachtete Anforderung (aktuell) nicht umgesetzt wird.
Die Priorisierung von Anforderungen ist nicht endgültig, sondern kann im Projektverlauf angepasst werden.
	}
}
\newglossaryentry{git} 
{
	name={Git-Repository},
	description={GIT ist ein Werkzeug zur Versionskontrolle, dass vor allem zur kollaborativen Quellcode-Verwaltung in Software-Projekten eingesetzt wird, sich aber auch zur Verwaltung und Versionierung von Artefakten der Dokumentation eignet.
Die von Microsoft betriebene Plattform GitHub.com bietet einen Dienst für das Hosting von Git-Repositories.}
}
\newglossaryentry{responsive}
{
	name={Responsive Design},
	description={Ist eine Applikation so gestaltet, dass sie sich an verschiedene Bildschirmgrößen und -ausrichtungen, wie sie beispielsweise bei Smartphones-, Tablets oder Desktop-PCs zu finden sind, anpassen kann, so spricht man von \textit{Responsive Design}.
	}
}
\newglossaryentry{uml}
{
	name={UML},
	description={Die \textit{Unified Modeling Language} ist eine grafische Modellierungssprache zur Analyse, Implementation und zum Design von softwarebasierten Systemen sowie zur Beschreibung von Prozessen. \autocite{UML} Sie wird durch die \textit{Object Management Group} entwickelt. Die Sprache definiert jeweils sieben Struktur- und Verhaltensdiagramme.
	In diesem Bericht werden folgende Diagrammtypen genutzt: 	\textit{Komponentendiagramm (cod)},\textit{ Klassendiagramm (cld)}, \textit{Zustandsdiagramm (sm)} und \textit{Sequenzdiagramm (sqd)}.
	}
}
\newglossaryentry{bpmn}
{
	name={BPMN},
	description={Die \textit{Business Process Model and Notation} ist ein Standard zur grafischen Spezifikation von Geschäftsprozessen. \autocite{BPMN} 
	}
}
\newglossaryentry{erm}
{
	name={ERM},
	description={Ein \textit{Entity-Relationship-Model} ist ein Modell zur Darstellung von Entitäten und Beziehungen. Der Einsatz von ERM gilt als Standard bei der Datenmodellierung in der Softwareentwicklung.
	}
}
\newglossaryentry{gui}
{
	name={GUI},
	description={Grafische Benutzeroberfläche
	}
}

\newglossaryentry{ssltls}
{
	name={SSL/TLS},
	description={Secure Socket Layer / Transport Layer Security. Verfahren zur Verschlüsselung von Netzwerkverbindungen.
	}
}

\newglossaryentry{ssh}
{
	name={SSH},
	description={Secure Shell. Verschlüsseltes Netzwerkprotokoll zum Konsolenzugriff auf Serversysteme.
	}
}

\subject{Meilensteinbericht}
\title{Meilenstein 3}
\subtitle{Dokumentationskonzept bereitgestellt}

\begin{document}
	\pagenumbering{Roman}
	\begin{titlepage}
		
		\centering
		\vspace*{2.5cm}
		{\large\bfseries \par}	
		{\Huge\bfseries Ergebnisdokumentation\par}
		{\Large\bfseries  \par}

		{\Large Projekt Q-Teams\par}
		{\large\today\par}
		\vspace{0.5cm}

			
		
		\includegraphics[scale=0.5]{iubh_logo}
		
		IUBH Fernstudium
		\vspace{0.5cm}
		
		\begin{tabular}{lllrl}
			\toprule
			\textbf{Gruppe} & \textbf{Nachname} & \textbf{Vorname} & \textbf{Matrikelnr.} & \textbf{Studiengang} \\
			\midrule
			Projektleiter & Sawatzki & Jörg & 9186524 & BSc. Informatik \\
			Mitglied 2 & Hahn & Maximilian & 91710055 & BSc. Wirtschaftsinformatik \\
			Mitglied 3 & Lapenat & Holger & 3191237 & BSc. Wirtschaftsinformatik \\
			Mitglied 4 & Moch & Daniel & 91710824 & BSc. Wirtschaftsinformatik \\
			\bottomrule
		\end{tabular}	
	\end{titlepage}
	
	
	\newpage
	\setcounter{tocdepth}{2}
	\tableofcontents	
	\renewcommand \thechapter{\Roman{chapter}}
	\listoffigures % ABBILDUNGSVERZEICHNIS
	\printglossaries
	\newcounter{lastRomanCounter}
	\setcounter{lastRomanCounter}{\value{chapter}} % Zwischenspeicher der Kapitelnummerierung (römisch)
	
	\newpage
	\renewcommand \thechapter{\arabic{chapter}}
	\pagenumbering{arabic}	
	\setcounter{chapter}{0}
	
	\chapter{Einleitung}
	Hier ein Test \cite{Prince2} und \Gls{ssh}.
	
	\chapter{Funktionsbeschreibung}
	
	\chapter{Softwarearchitektur}
	\section{Systemstruktur}
	\section{Daten}
	\section{GUI}
	\section{Komponenten}
	\section{Zeit}
	
	

	\setcounter{chapter}{0}
	
	
	
	% ANHANG
	\begin{appendices}
	\renewcommand{\appendixtocname}{Anhang}
	\renewcommand{\appendixpagename}{Anhang}
	\appendixpage	
	\addappheadtotoc
	\chapter{Anforderungen}	
		
		\section{Funktionale Anforderungen}
		
	\subsection{Als Spieler möchte ich mit dem Quizsystem gezielt fachliche Inhalte wiederholen, um mich auf
Prüfungen vorzubereiten.}

		\begin{tabularx}{\textwidth}{Xr}
			
			Kriterien & Erfüllungsgrad \\
			\midrule
		Das Thema einer Quizrunde kann vor Beginn festgelegt werden. & \\
		Die Module der IUBH können als Thema ausgewählt werden. & \\
		Die Lernzyklen der Module können als Thema ausgewählt werden. & Nicht umgesetzt \\
			\bottomrule
		\end{tabularx}		
		
		\subsection{Als Spieler möchte ich gegen einen anderen Spieler antreten können, um mich mit ihm zu messen /
vergleichen.}

		\begin{tabularx}{\textwidth}{Xr}
			
			Kriterien & Erfüllungsgrad \\
			\midrule
		Das Spiel kann mit anderen Menschen gespielt werden. & \\
		Es kann ein kompetitiver Spielmodus festgelegt werden. & \\
		Den Spielern werden ceteris paribus die gleichen Fragen gestellt. & \\		
		Die Antworten werden mit Punkten bewertet. & \\
		Den Spielern wird eine Rangliste anhand ihrer Punkte angezeigt. & \\
			\bottomrule
		\end{tabularx}	
		
		\subsection{Als Spieler möchte ich mit einem anderen Spieler gemeinsam spielen, um mich im Team über Lerninhalte austauschen zu können.}	
			\begin{tabularx}{\textwidth}{Xr}
			
			Kriterien & Erfüllungsgrad \\
			\midrule
		Das Spiel kann mit anderen Menschen gespielt werden. & \\
		Es kann ein Team gebildet werden. & \\
		Es kann ein kooperativer Spielmodus festgelegt werden. & \\
			\bottomrule
		\end{tabularx}	
		
		\subsection{Als Spieler möchte ich Fragen und Antworten für andere Spieler hinzufügen und pflegen, um mein
Wissen weiterzugeben und zu festigen.}
		\begin{tabularx}{\textwidth}{Xr}
			
			Kriterien & Erfüllungsgrad \\
			\midrule
		Fragen und Antworten können persistiert werden. & \\
		Persistierte Fragen und Antworten können angepasst werden. & \\
		Persistierte Fragen und Antworten können gelöscht werden. & \\
		Persistierte Fragen und Antworten können angesehen werden. & \\
		Persistierte Fragen und Antworten können in einer Spielrunde genutzt werden. & \\
			\bottomrule
		\end{tabularx}		
		
		\subsection{Als Spieler möchte ich Feedback zu meiner Lösung bekommen, um aus Fehlern zu lernen.}
		\begin{tabularx}{\textwidth}{Xr}
			
			Kriterien & Erfüllungsgrad \\
			\midrule
			Die Antworten werden bewertet. & \\
			Dem Spieler wird die Bewertung der Antwort angezeigt. & \\
			
			\bottomrule
		\end{tabularx}	
		
		\section{Nicht-funktionale Anforderungen}
		
		\subsection{Als Spieler möchte ich das Quiz im Browser spielen können, damit ich es auf verschiedenen
Plattformen spielen kann und keine Zusatzsoftware installieren muss.}
		\begin{tabularx}{\textwidth}{Xr}
			
			Kriterien & Erfüllungsgrad \\
			\midrule
		\textit{Spielen} bedeutet hier den Ablauf einer Spielrunde in den Modi \textit{kompetitiv} und \textit{kooperativ} in einem Team mit 5 Spielern ohne Spielabbrüche durchzuspielen. 	
		Das Spiel ist mit folgenden Browsern in der Standardinstallation spielbar: & \\
		Quizrunde mit folgenden Browsern spielen: & \\
		IE 11.778.18362.0 & \\
		Firefox 76.0 & \\
		Edge 81.0.416.68 & \\
		Chrome 81.0.4044.138 & \\
			\bottomrule
		\end{tabularx}	
			
		\subsection{Als Auftraggeber wünsche ich mir einen lauffähigen Prototyp, damit ich die mögliche Einsatzfähigkeit in der Praxis evaluieren kann.}
		\begin{tabularx}{\textwidth}{Xr}
			
			Kriterien & Erfüllungsgrad \\
			\midrule
			Es kann im Browser Firefox 76.0 in einem Team mit 3 Spielern jeweils eine Spielrunde in den Modi \textit{kompetitiv} und \textit{kooperativ} durchgespielt werden. & \\
			Es kann im Browser Firefox 76.0 in der Erarbeitungsphase eine Frage mit Antwort persistiert werden. & \\
			Es können im Browser Firefox 76.0 in der Erarbeitungsphase die bereits persistierten Fragen mit Antworten angezeigt werden. & \\
			
			\bottomrule
		\end{tabularx}	
				
		\subsection{Als Auftraggeber / Spieler wünsche ich mir ein Benutzerhandbuch, um mir einen Überblick über
Aufbau und Funktionen des Systems verschaffen zu können.}
		\begin{tabularx}{\textwidth}{Xr}
			
			Kriterien & Erfüllungsgrad \\
			\midrule
		Das Benutzerhandbuch wird dem Auftraggeber zugestellt. & \\
		Das Benutzerhandbuch steht einem Spieler in dem System zur Verfügung. & \\
		Jede GUI-Maske wird mit einem Screenshot dargestellt und erläutert. & \\
		Die Spielregeln werden erklärt. & \\
			\bottomrule
		\end{tabularx}	
		
		\subsection{Als Auftraggeber / Betreiber wünsche ich mir eine Dokumentation zur Installation und
Inbetriebnahme, damit ich das Quiz ohne großen Aufwand den Studenten zur Verfügung stellen
kann.}
		\begin{tabularx}{\textwidth}{Xr}
			
			Kriterien & Erfüllungsgrad \\
			\midrule
		Die Dokumentation wird dem Auftraggeber zugestellt. & \\
		Die Dokumentation steht einem potenziellen Betreiber zur Verfügung. & \\
		Die Dokumentation beschreibt die notwendige Hardware zur Installation und Inbetriebnahme. & \\
		Die Dokumentation beschreibt die notwendige Software zur Installation und Inbetriebnahme. & \\
		Die Dokumentation beschreibt die notwendigen Schritte zur Installation und Inbetriebnahme. & \\
		Die Dokumentation beschreibt mögliche Fehlerfälle mit den dazugehörigen Lösungen. & \\
		
			\bottomrule
		\end{tabularx}	

		
			
		
		
		
	
	\chapter{Qualitätsziele}
	\chapter{Benutzerhandbuch}
	\chapter{Administration}
	\chapter{Inbetriebnahme}
	
	\end{appendices}
	\newpage	
	\setcounter{chapter}{\thelastRomanCounter} % Fortsetzen der römischen Kapitelnummerierung
\renewcommand \thechapter{\Roman{chapter}}	\printbibliography[heading=bibnumbered,title=Literaturverzeichnis]

\end{document}
